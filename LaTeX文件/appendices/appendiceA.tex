\section{更换软件源为清华大学镜像站}
\label{sec:appendix-ubuntu-mirror}

\subsection{Ubuntu 24.04.2 LTS 软件源配置}
如果是按照文章开始的方式安装的Ubuntu系统,安装的版本应当是24.04.2 LTS(长期支持版),
请确保版本是24.04.2。
可以通过以下命令查看当前系统版本:
\begin{envcode}{console}{Bash}
user1@host:~$ lsb_release -a
No LSB modules are available.
Distributor ID: Ubuntu
Description:    Ubuntu 24.04.2 LTS
Release:        24.04
Codename:       noble
\end{envcode}

运行如下命令,打开软件源配置文件:
\begin{envcode}{console}{Bash}
user1@host:~$ sudo vim /etc/apt/sources.list.d/ubuntu.sources
\end{envcode}

不断按\code{dd}删除原有内容,直到清空文件。

然后按\code{i}进入插入模式,粘贴以下内容:
\begin{envcode}{text}{Text}
Types: deb
URIs: https://mirrors.tuna.tsinghua.edu.cn/ubuntu
Suites: noble noble-updates noble-backports
Components: main restricted universe multiverse
Signed-By: /usr/share/keyrings/ubuntu-archive-keyring.gpg

# 默认注释了源码镜像以提高 apt update 速度,如有需要可自行取消注释
# Types: deb-src
# URIs: https://mirrors.tuna.tsinghua.edu.cn/ubuntu
# Suites: noble noble-updates noble-backports
# Components: main restricted universe multiverse
# Signed-By: /usr/share/keyrings/ubuntu-archive-keyring.gpg

# 以下安全更新软件源包含了官方源与镜像站配置,如有需要可自行修改注释切换
Types: deb
URIs: http://security.ubuntu.com/ubuntu/
Suites: noble-security
Components: main restricted universe multiverse
Signed-By: /usr/share/keyrings/ubuntu-archive-keyring.gpg

# Types: deb-src
# URIs: http://security.ubuntu.com/ubuntu/
# Suites: noble-security
# Components: main restricted universe multiverse
# Signed-By: /usr/share/keyrings/ubuntu-archive-keyring.gpg

# 预发布软件源,不建议启用

# Types: deb
# URIs: https://mirrors.tuna.tsinghua.edu.cn/ubuntu
# Suites: noble-proposed
# Components: main restricted universe multiverse
# Signed-By: /usr/share/keyrings/ubuntu-archive-keyring.gpg

# # Types: deb-src
# # URIs: https://mirrors.tuna.tsinghua.edu.cn/ubuntu
# # Suites: noble-proposed
# # Components: main restricted universe multiverse
# # Signed-By: /usr/share/keyrings/ubuntu-archive-keyring.gpg
\end{envcode}

按“Esc”键退出插入模式,然后输入\code{:wq}后回车保存并退出。

接下来,运行以下命令更新软件源列表:
\begin{envcode}{console}{Bash}
user1@host:~$ sudo apt-get update
\end{envcode}

\subsection{更早或更新版本软件源配置}

清华大学开源软件镜像站一般只有LTS版本的Ubuntu软件源,首先请确保自己系统为LTS版本。

访问\textbf{\textcolor{blue}{\href{https://mirrors.tuna.tsinghua.edu.cn/help/ubuntu/}{清华大学开源软件镜像站}}},
仔细阅读网页指导内容,仿照上例更改软件源配置。或自行查找相关教程,这里不作介绍。

