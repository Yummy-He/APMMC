\section{写在前面}

\subsection{章节结构}
本笔记层级为章节、小节、小小节,例如1、1.1、1.1.1等。

笔记中代码块以小节为基础,向下编号,与小小节并行,与小小节编号无关。如:
\begin{envcode}{text}{text}
这是第0节第1小节第1个代码块。
\end{envcode}
\begin{envcode}{text}{text}
这是第0节第1小节第2个代码块。
\end{envcode}

笔记中总结以章节为基础,向下编号,与小节并行,与小节编号无关。如:
\begin{zj}
这是第0节第1小节第1个总结。
\end{zj}
\begin{zj}
这是第0节第1小节第2个总结。
\end{zj}

\subsection{字体}

本笔记正文中文字体为宋体,英文字体为Times New Roman。

标题字体为思源宋体,页眉中文为楷体,页眉英文为Times New Roman。

代码块中字体为Maple字体。
Maple字体中,字符等宽、显示清晰,需要注意的是\code{@}为“@”符号。

自己需要在PPT或Word等软件中展示代码时,推荐使用电脑自带的Consolas字体,是英文等宽字体。
本笔记为了实现中英文均等宽(一个中文字符严格占两个英文字符空间),使用了Maple字体。
\begin{envcode}{text}{text}
一二三四五六七八九十百千万
AbCdEfGhIjKlMnOpQrStUvWxYz
\end{envcode}

\subsection{内容}
本笔记基于课堂PPT内容整理。对于超出课程范围的内容,笔记中会标出,并且可能会有错误或不准确之处。

课程范围内的内容,笔记中力求准确、清晰、易懂,但水平所限,难免有疏漏或错误之处,注意甄别,并请不吝指出。