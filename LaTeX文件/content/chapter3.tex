\newpage
\section{Fortran基础}
\subsection{Fortran变量类型声明与转化}
\subsubsection{Fortran变量类型声明}
Fortran语言中,变量的类型和声明是非常重要的概念。
Fortran支持多种数据类型,包括整数、实数(与数学上的实数定义不同)、字符等。

变量名可以包含字母、数字和下划线,但要以字母开头,同时不超过6个字符。

Fortran有隐式声明和显式声明两种方式。

隐式声明是指在使用变量之前不需要显式地声明变量的类型,Fortran会根据变量名的首字母自动推断类型,
例如以\code{I、J、K、L、M、N}开头的变量默认为整数类型,其它变量默认为实数类型。

显式声明是指在使用变量之前需要显式地声明变量的类型,使用类型声明语句,例如:
\begin{envcode}{fortran}{Fortran}
      INTEGER :: I
      REAL :: X
      DOUBLE PRECISION :: Y
\end{envcode}
上面第一行声明了一个整数类型的变量\code{I},第二行声明了一个单精度实数类型的变量\code{X},第三行声明了一个双精度实数类型的变量\code{Y}。

一般我们遵循隐式声明的规则,除非有特殊需要,否则不使用显式声明。

\subsubsection{Fortran变量类型转化}
Fortran语言中,变量类型的转化是一个重要的概念。Fortran提供了多种方式来进行变量类型的转化。

使用类型转换函数,例如:
\begin{envcode}{fortran}{Fortran}
      M = INT(A)
      N = IFIX(B)
      I = IDINT(C)
\end{envcode}
前两个语句将实数\code{A}和\code{B}转换为整数类型,第三个语句将双精度实数\code{C}转换为整数类型。可统一使用\code{INT}函数进行转换。

\begin{envcode}{fortran}{Fortran}
      X = REAL(I)
      Y = FLOAT(J)
      Z = DBLE(K)
\end{envcode}
前两个语句作用相同,将整数\code{I}和\code{J}转换为单精度实数类型,第三个语句将整数K转换为双精度实数类型。

以上可统一记为\code{INT}转化为整数,\code{REAL}转化为单精度实数,\code{DBLE}转化为双精度实数,其它的都能用这三个代替。

\subsection{Fortran程序的输入与输出}

Fortran程序的输入与输出主要通过以下几个语句实现:

\begin{itemize}
    \item \code{READ}:用于从标准输入设备(通常是键盘)读取数据。
    \item \code{WRITE}:用于向标准输出设备(通常是屏幕)输出数据。
    \item \code{PRINT}:用于向标准输出设备写入数据,语法更简单。
\end{itemize}

我们主要使用\code{READ}和\code{WRITE}语句来进行输入和输出。

\subsubsection{输入语句}

Fortran中的输入语句主要有两种形式:\code{READ}和\code{READ(*,*)}。

READ语句的基本形式为:
\begin{envcode}{fortran}{Fortran}
      READ, 变量
      READ(单位, 格式) 变量1 变量2 ...
\end{envcode}

其中,单位表示输入设备,其中\code{5}表示读卡器(在现代的电脑上就是用键盘输入),格式表示数据的格式,变量列表表示要读取的变量。

\code{READ(5,*)}表示从标准输入设备读取数据,格式由编译器自动推断,适用于简单的输入场景。

\subsubsection{输出语句}

Fortran中的输出语句主要有两种形式:\code{WRITE}和\code{PRINT}。


WRITE语句的基本形式为:
\begin{envcode}{fortran}{Fortran}
      WRITE(单位, 格式) 变量1 变量2 ...
\end{envcode}
其中,单位表示输出设备,其中\code{6}表示行式打印机(在现代的电脑上就是用屏幕输出),格式表示数据的格式,变量列表表示要输出的变量。

\texttt{PRINT}语句的基本形式为:
\begin{envcode}{fortran}{Fortran}
      PRINT(格式) 变量1 变量2 ...
\end{envcode}
    其中,格式表示数据的格式,变量列表表示要输出的变量。








\subsection{格式化输出}

Fortran支持格式化输入输出,可以通过格式字符串来控制输出的格式。
一般情况下,输入数据来源复杂,格式多样,我们会使用\code{*}表示自动推断格式;
输出时为了获取整齐美观的结果,可以指定格式。
下面只介绍输出的格式化,输入格式化类似,不再赘述。

\subsubsection{格式化输出语句}

上节命令如\code{READ}和\code{WRITE}中都有“格式”参数,一般读取时我们用\code{*}表示自动推断格式,
输出时我们可以指定格式。

指定格式有两种方式:

第一种是使用格式字符串,例如:
\begin{envcode}{fortran}{Fortran}
      WRITE(6, '(A, I5, F10.2, E10.2)') '结果是:', I, X
\end{envcode}
其中,\code{'(A, I5, F10.2, E10.2)'}表示输出一个字符串(A),一个整数(I5,宽度为5),一个单精度实数(F10.2,宽度为10,小数点后保留2位),一个双精度实数(E10.2,宽度为10,小数点后保留2位)。

第二种是使用格式标签,例如:
\begin{envcode}{fortran}{Fortran}
      WRITE(6, 100) I, X
  100 FORMAT('结果是:', I5, F10.2, E10.2)
\end{envcode}
其中,\code{100}是标签,其指定的行中的\code{FORMAT}语句定义了格式。如果格式字符串较长,可以使用标签来分割,避免超出固定格式行长度限制。

\subsubsection{I型}
I型格式用于输出整数,格式为\code{Iw},其中\code{w}表示输出的宽度,
可以理解为给这个数据5个字符的空间,例如:
\begin{envcode}{fortran}{Fortran}
      M=12345
      N=123
      K=123456
      WRITE(6, '(I5)') M
      WRITE(6, '(I5)') N
      WRITE(6, '(I5)') K
      END
\end{envcode}
编译后运行输出结果为:
\begin{envcode}{text}{Text}
12345
  123
*****
\end{envcode}
其中,\code{*****}表示输出的整数超过了指定的宽度,Fortran会用星号填充。

\subsubsection{F型}
F型格式用于输出单精度实数,格式为\code{Fw.d},
其中\code{w}表示输出的宽度,\code{d}表示小数点后保留的位数,
例如:
\begin{envcode}{fortran}{Fortran}
      K=1234567890
      X=123.456
      WRITE(6, '(I10)') K
      WRITE(6, '(F10.2)') X
      END
\end{envcode}
编译后运行输出结果为:
\begin{envcode}{text}{Text}
1234567890
    123.46
\end{envcode}
(这里第一行的整数用来标记位置,便于直观观察F型所占空间)
其中,\code{123.46}表示输出的单精度实数,Fortran会根据指定的格式进行填充和舍入。

\subsubsection{E型}
E型格式用于输出双精度实数,格式为\code{Ew.d},
其中\code{w}表示输出的宽度,\code{d}表示小数点后保留的位数,
例如:
\begin{envcode}{fortran}{Fortran}
      K=1234567890
      Y=-123.456
      Z=123.456
      WRITE(6, '(I10)') K
      WRITE(6, '(E10.3)') Y      
      WRITE(6, '(E10.3)') Z
      END
\end{envcode}
编译运行后输出结果为:
\begin{envcode}{text}{Text}
1234567890
-0.123E+03
 0.123E+03
\end{envcode}
(这里第一行的整数用来标记位置,便于直观观察E型所占空间)
其中,\code{0.123E+03}表示输出的双精度实数。

Fortran中标准的E型格式需要注意,一个数用科学计数法表示为\code{A*10^N},
表示为E型为\code{AEN},标准化E型中,A的整数部分为\code{0}。
如\code{123.456},则表示为\code{0.123456E+03}而不是\code{1.23456E+02},再根据格式进行填充和舍入。

以上内容可见,E型占用的空间较大,使用时注意前后留足空间,如使用\code{E20.5}。



\subsection{数学语句}

\subsubsection{数字与运算符}

在编写Fortran程序时,我们需要使用数字和运算符来进行计算。
整数如\code{123}、\code{-456}等,直接表示即可。

实数如\code{3.14}、\code{-2.718}等,直接表示即可;
\code{0.123}、\code{456.0},可以省略0,直接写为\code{.123}、\code{456.}。

Fortran常用的运算符有加(\code{+})、减(\code{-})、乘(\code{*})、除(\code{/})和幂(\code{**})等。
可直接对数字或变量进行运算。
运算优先级从高到低依次为:幂、乘除、加减。如果有括号,则括号内的运算优先级最高。

\subsubsection{数学函数}
Fortran提供了多种数学函数,用于对数字进行处理和计算。
例如:
\begin{envcode}{fortran}{Fortran}
      A = ABS(X)       ! 绝对值
      B = SQRT(Y)      ! 平方根
      C = EXP(Z)       ! 指数函数
      D = LOG(W)       ! 自然对数
      E = LOG10(V)     ! 常用对数
      F = SIN(T)       ! 正弦函数
      G = COS(U)       ! 余弦函数
      M = MAX(A, B)    ! 最大值
      N = MIN(C, D)    ! 最小值
      O = MOD(P, Q)    ! 取模
\end{envcode}

其中自然对数函数在课堂上介绍的是\code{ALOG},这是Fortran 77的函数,用于处理单精度实数;
另外还有\code{DLOG}用于处理双精度实数、\code{CLOG}用于处理复数,比较复杂。

在Fortran 90中,增加了\code{LOG}函数,包括了上面三种用法,兼容性强,推荐使用。

\subsubsection{赋值语句}
赋值语句用于将一个值赋给一个变量,例如:
\begin{envcode}{fortran}{Fortran}
      I = 5
      X = 3.14
      Y = 2.718281828459045
\end{envcode}

\subsubsection{算术运算}
Fortran支持基本的算术运算,包括加法、减法、乘法、除法等,例如:
\begin{envcode}{fortran}{Fortran}
      Z = X + Y
      A = B - C
      D = E * F
      G = H / I
      J = K ** L
\end{envcode}

\subsubsection{大型算术表达式}
Fortran支持大型算术表达式,可通过嵌套多层括号、合理利用运算符的优先级来实现复杂的计算。
例如:
\begin{equation*}
      z = \frac{\mathrm{e}^{x+y}+\sin{x}+s^t}{\left|x\right| + \sqrt{y} - \lg(w)}
\end{equation*}
用Fortran的数学语句表示为:
\begin{envcode}{fortran}{Fortran}
      Z = (EXP(X+Y) + SIN(X) + S**T) / (ABS(X) + SQRT(Y) - LOG10(W))
\end{envcode}

为了防止语句过长违反固定格式,可拆开表示:
\begin{envcode}{fortran}{Fortran}
      P = EXP(X+Y) + SIN(X) + S**T
      Q = ABS(X) + SQRT(Y) - LOG10(W)
      Z = P / Q
\end{envcode}

\subsection{Fortran程序构成}
Fortran程序一般由以下几个部分构成:
\begin{itemize}
      \item 程序头:包括程序名、变量声明等。
      \item 主程序:包含程序的主要逻辑和计算部分。
      \item 子程序:可选部分,用于实现特定功能的代码块。
      \item 结束语句:标志程序的结束。
\end{itemize}
例如:
\begin{envcode}{fortran}{Fortran}
      PROGRAM HelloWorld
      INTEGER :: I
      REAL :: X, Y
      
      WRITE(6, '(A)') 'Hello, World!'
      
      I = 5
      X = 3.14
      Y = 2.718281828459045
      
      WRITE(6, '(I5, F10.2, E10.3)') I, X, Y
      
      END PROGRAM HelloWorld
\end{envcode}

其中第一行\code{PROGRAM HelloWorld}表示程序名为\code{HelloWorld},非必须,可以不写。

接下来是变量声明部分,使用\code{INTEGER}和\code{REAL}声明了整数和实数类型的变量,这里的内容实际上包含在隐式声明内了,可以不再显式声明。

\code{END}语句标志程序的结束,后面的\code{PROGRAM HelloWorld}表示程序名的结束,可以直接写\code{END}。

测试。
